% $Header$

\documentclass{beamer}

% This file is a solution template for:

% - Giving a talk on some subject.
% - The talk is between 15min and 45min long.
% - Style is ornate.



% Copyright 2004 by Till Tantau <tantau@users.sourceforge.net>.
%
% In principle, this file can be redistributed and/or modified under
% the terms of the GNU Public License, version 2.
%
% However, this file is supposed to be a template to be modified
% for your own needs. For this reason, if you use this file as a
% template and not specifically distribute it as part of a another
% package/program, I grant the extra permission to freely copy and
% modify this file as you see fit and even to delete this copyright
% notice. 


\mode<presentation>
{
  \usetheme{Berkeley}

  % Code syntax highlighter
  \usepackage{minted}
  
  \usecolortheme{dove}
  % or ...

  \definecolor{mainnetGreen}{rgb}{0.24, 0.68, 0.483} % 

  % \usecolortheme[named=mainnetGreen]{structure}
   \setbeamercolor{palette primary}{bg=mainnetGreen,fg=white}
   \setbeamercolor{palette secondary}{bg=mainnetGreen,fg=white}
   \setbeamercolor{palette tertiary}{bg=mainnetGreen,fg=white}
   \setbeamercolor{palette quaternary}{bg=mainnetGreen,fg=white}


  \setbeamercovered{transparent}
  % or whatever (possibly just delete it)
}


\usepackage[english]{babel}
% or whatever

\usepackage[latin1]{inputenc}
% or whatever


\usepackage{hyperref}
\hypersetup{colorlinks=true,allcolors=blue}

\usepackage[T1]{fontenc}
% Or whatever. Note that the encoding and the font should match. If T1
% does not look nice, try deleting the line with the fontenc.


\title[smartBCH] % (optional, use only with long paper titles)
{Using SmartBCH via Mainnet Cash}

\subtitle
{A Quick Start} % (optional)

\author[] % (optional, use only with lots of authors)
{2qx - github.com/2qx}

\date[HackSmart] % (optional)
{2022-06-20 / HackSmart Dojo}

\subject{Using SmartBCH with mainnet.cash}
% This is only inserted into the PDF information catalog. Can be left
% out. 



% If you have a file called "university-logo-filename.xxx", where xxx
% is a graphic format that can be processed by latex or pdflatex,
% resp., then you can add a logo as follows:

% \pgfdeclareimage[height=0.5cm]{university-logo}{university-logo-filename}
% \logo{\pgfuseimage{university-logo}}



% Delete this, if you do not want the table of contents to pop up at
% the beginning of each subsection:
% \AtBeginSubsection[]
% {
%   \begin{frame}<beamer>{Outline}
%     \tableofcontents[currentsection]
%   \end{frame}
% }


% If you wish to uncover everything in a step-wise fashion, uncomment
% the following command: 

%\beamerdefaultoverlayspecification{<+->}


\begin{document}

\begin{frame}
  \titlepage
\end{frame}

\begin{frame}{Outline}
  \tableofcontents
  % You might wish to add the option [pausesections]
\end{frame}


% Since this a solution template for a generic talk, very little can
% be said about how it should be structured. However, the talk length
% of between 15min and 45min and the theme suggest that you stick to
% the following rules:  

% - Exactly two or three sections (other than the summary).
% - At *most* three subsections per section.
% - Talk about 30s to 2min per frame. So there should be between about
%   15 and 30 frames, all told.

\section{Intro}

\subsection[What You Need]{Pre-installed Software}

\begin{frame}{Suggested Tooling}{To Follow Along On Your Own Battle Station}
  % - A title should summarize the slide in an understandable fashion
  %   for anyone how does not follow everything on the slide itself.

  

  \begin{itemize}
    \item
    \href{https://raw.githubusercontent.com/2qx/mainnet-tutorial/main/slides/smarthBCH_quickstart.pdf}{These slides}: https://bit.ly/3MZEXJq
    \item
    A computer with a POSIX command prompt and some GNU tooling, i.e. (Linux/WSL/Darwin)
    \item   
    \href{https://docs.docker.com/get-docker/}{Docker} \& \href{https://docs.docker.com/compose/install/compose-desktop/}{docker-compose}
    \item
    The \texttt{mainnet-js} \href{https://github.com/mainnet-cash/mainnet-js/}{source code} \\
    \texttt{git clone https://github.com/mainnet-cash/mainnet-js.git; cd mainnet-js}
    \item
    Run \texttt{yarn regtest:sbch:up} from the mainnet-js root directory to pre-download \& deploy a RegTest network stack.
    \item
    Some kind of nodejs, or python http serverlet \\
    \texttt{python3} or \texttt{npm i reload -g}
  \item
    Browser, with Metamask installed
  \end{itemize}
\end{frame}

\subsection[Mainnet Cash]{What is Mainnet Cash?}

\begin{frame}{Mainnet Cash is a Specification}{First and foremost.}
  % - A title should summarize the slide in an understandable fashion
  %   for anyone how does not follow everything on the slide itself.

  Mainnet Cash is a tool for developers to access Bitcoin Cash
  \begin{itemize}
  \item
    A contract for how a service will be provided.
  \item
    Generated and enforced to the specification programmatically, to the extent practical
  \item
    The specification: \href{https://rest-unstable.mainnet.cash/api-docs/}{api.yml}
  \item
    Specifically, an OpenAPI 3 spec, implement in typescript, used in a generated express server. 
  \end{itemize}
  Developer convenience and expediency are valued over perfection, code cleanness or symmetry.
\end{frame}

\section{Dev Harness}

\subsection[Docker image]{Using the mainnet.cash Docker image}

\begin{frame}[fragile]
    \frametitle{Starting your own mainnet.cash}
      To start your own mainnet.cash REST service
    \rule{\textwidth}{0.9pt}
    \tiny
    \begin{minted}{shell}
# get the image from docker
docker pull mainnet/mainnet-rest

# start the image, with one process, 
# exposing locally at 3000 to the container's port 80
# & pass it details of the regression configuration
docker run -d --name Mainnet --env WORKERS=1 \ 
            -p 127.0.0.1:3000:80 \ 
            --env-file .env.regtest \
            mainnet/mainnet-rest
    \end{minted}
\rule{\textwidth}{0.9pt}

\end{frame}

\begin{frame}[fragile]
    \frametitle{Stopping your own mainnet.cash}
      Just in case something \alert{goes wrong}...
    \rule{\textwidth}{0.6pt}
    \tiny
    \begin{minted}{shell}
# checking logs, to find the issue
docker logs -f Mainnet

# restarting, stopping, forcefully stopping.
docker restart Mainnet
docker stop Mainnet
docker kill Mainnet

# if your machine is paging memory or sounds like a jet plane
sudo killall node 
# kills all docker processes
docker stop $(docker ps -a -q)
    \end{minted}
\rule{\textwidth}{0.6pt}
assuming your container was given the name "Mainnet".\\
\end{frame}

\begin{frame}[fragile]
    \frametitle{More Usage Samples and Configuration Information}
  % Keep the summary *very short*.
  \begin{itemize}
    \item
      There's more at: \href{https://mainnet.cash/tutorial/running-rest.html}{mainnet.cash/tutorial/running-rest.html}
    \item
      Including a full list of \href{https://mainnet.cash/tutorial/running-rest.html#configuration}{configuration variables.}
    \item
      This is an \href{https://rest-unstable.mainnet.cash/api-docs/}{unstable version} of the REST service for testing and educational purposes
    \end{itemize}
        
\rule{\textwidth}{0.9pt}
\end{frame}

\subsection[RegTest Stack]{Spin-up a Full Service Stack in Regtest}

\begin{frame}[fragile]
    \frametitle{Running a RegTest Stack}
      The mainnet-js source code contains a full RegTest Stack, deployed with docker compose.
    \rule{\textwidth}{0.6pt}
    \tiny
    \begin{minted}{shell}

#  To start a regtest stack,
# from the mainnet-js source directory...
yarn regtest:sbch:up

# reset everything
yarn regtest:sbch:down; yarn regtest:sbch:up
    \end{minted}
\rule{\textwidth}{0.6pt}
There's more in the \href{http://mainnet.cash/tutorial/#regtest-wallets}{docs}.\\

SmartBCH has \href{https://docs.smartbch.org/smartbch/developers-guide/runsinglenode}{alternative documentation} 

Having your own private networks allow for better tests, testing in time, testing with a RegTest Bitcoin Cash chain. \\ 
Note: the docker images for the stack took about 6 min to download over wifi.
\end{frame}

\begin{frame}[fragile]
    \frametitle{The RegTest Secrets}
    One of the advantages of RegTest is reusing and sharing secrets  
    \rule{\textwidth}{0.4pt}
    \tiny
    \begin{minted}{javascript}
ALICE_ID="privkey:regtest:0x758c7be51a76a9b6bc6b3e1a90e5ff4cc27aa054b77b7acb6f4f08a219c1ce45";
ALICE_ADDRESS="0xE25ddbAF8DD61b627727e03e190E32feddBD1319";
BOB_ID="seed:regtest:"+
"total unlock silver actual hunt excuse tone hen fix turtle addict bracket:"+
"m/44'/60'/0'/0/0";
BOB_ADDRESS="0xaf9aD25ea3603B0aF520e1B5331d87315B6eB72B";
    \end{minted}
\rule{\textwidth}{0.4pt}
\end{frame}





\subsection[Javascript]{Running mainnet-js Bundled Script}

\begin{frame}[fragile]
    \frametitle{Run the packages from a Browser}
    When developing from a jsfiddle or codepen...
    \rule{\textwidth}{0.4pt}
    \tiny
    \begin{minted}{html}
<!-- mainnet-js -->
<script src="https://cdn.mainnet.cash/mainnet-0.5.4.js"
        integrity="sha384-W9XdLRFn3qb7qn1+gqFho+FTw9buULZBaQh+uceAAMmK3wKyJ/GC/3kcWUOpSitj"
        crossorigin="anonymous">
</script>
<!-- CashScript -->
<script src="https://cdn.mainnet.cash/contract/contract-0.5.4.js"
        integrity="sha384-4htLbqsKWOCH8CvR0ESHkSDw1lRiDXodGnw7e2xc3mcArviVp4neGsHKQKvEs7Wd"
        crossorigin="anonymous">

</script>
<!-- SmartBCH -->
<script src="https://cdn.mainnet.cash/smartbch/smartbch-0.5.4.js"
        integrity="sha384-sk/jV4NeaEsrfMShEiO3C8zMwEOHzwfrVu6s6EiVoqOmnU87RT7WRsSNxztaxi7v"
        crossorigin="anonymous">
</script>
    \end{minted}
\rule{\textwidth}{0.4pt}
\end{frame}


\begin{frame}[fragile]
    \frametitle{Run the packages from a Browser}
We want to load all the most recent bundles in a browser to access the library from the dev console...
    \rule{\textwidth}{0.4pt}
    \tiny
    \begin{minted}{shell}
# open a terminal and cd to the mainnet-js directory
cd projects/mainnet-js

# serve an empty app with the smartBCH bundle preloaded
python3 -m http.server -d jest/playwright/smartbch/

# python3 may be called python, 

    \end{minted}
\rule{\textwidth}{0.4pt}

\texttt{npx reload} or similar is fine too. The script \texttt{yarn live}  is a shortcut

You should now have a browser console with the latest bundles on \href{http://127.0.0.1:8000}{port 8000}. 
\\ Simply pull mainnet-js to update versions.
\end{frame}


\section{sBCH Wallets}

\begin{frame}[fragile]
    \frametitle{Get a wallet}
    If you \alert{didn't get RegTest} harness working... \\
    Create a TestNet Wallet and get the deposit address:
    \rule{\textwidth}{0.9pt}
    \tiny
    \begin{minted}{javascript}
const alice = await TestNetSmartBchWallet.newRandom();
// 
await alice.getBalance()
// Object { bch: 0, sat: 0, usd: 0 }
alice.getDepositAddress()
// "0xE25ddbAF8DD61b627727e03e190E32feddBD1319"
alice.toString()
// "seed:t........"
    \end{minted}
\rule{\textwidth}{0.9pt}

Save the wallet string, use it again with \texttt{TestNetSmartBchWallet.fromId()}
\end{frame}

\subsection[Getting Money]{Receiving From Faucet}

\begin{frame}[fragile]
    \frametitle{Getting Testnet sBCH}
If you are \alert{not} using RegTest, you can obtain about 0.1 sBCH here:
    \begin{itemize}
    \item
    There is a TestNet \href{http://54.169.31.93:8080/faucet}{faucet here}
    \item
    54.169.31.93:8080/faucet    
    \end{itemize}
On site users should skip this step if they have RegTest configured.
\end{frame}


\begin{frame}[fragile]
    \frametitle{Get a wallet}
    Create a RegTest Wallet and get the deposit address:
    \rule{\textwidth}{0.9pt}
    \tiny
    \begin{minted}{javascript}
ALICE_ID="privkey:regtest:0x758c7be51a76a9b6bc6b3e1a90e5ff4cc27aa054b77b7acb6f4f08a219c1ce45"
const alice = await RegTestSmartBchWallet.fromId(ALICE_ID);
// 
await alice.getBalance()
// Object { bch: 10000, sat: 1000000000000, usd: 1141000 }
alice.getDepositAddress()
// "0xE25ddbAF8DD61b627727e03e190E32feddBD1319"
alice.toString()
// ALICE_ID
    \end{minted}
\rule{\textwidth}{0.9pt}
\end{frame}




\subsection[Sending sBCH]{Send \& Watch for Reciept}
\begin{frame}[fragile]
    \frametitle{Sending/Watching}
    Watch Bob's address as Alice sends coins:
    \rule{\textwidth}{0.9pt}
    \tiny
    \begin{minted}{javascript}
ALICE_ID="privkey:regtest:0x758c7be51a76a9b6bc6b3e1a90e5ff4cc27aa054b77b7acb6f4f08a219c1ce45"
alice = await RegTestSmartBchWallet.fromId(ALICE_ID);
const bob = await RegTestSmartBchWallet.newRandom();

const bobBalanceWatchCancel = bob.watchBalance((balance) => {
    bobBalanceWatchCancel();
    console.log("bob got paid!")
});

alice.send(
    { address: bob.getDepositAddress(), value: 0.001, unit: "bch" },
    {},
    { gasPrice: 10 ** 10 }
);
    \end{minted}
\rule{\textwidth}{0.9pt}
\end{frame}





\section[Web3 Wallets]{Web3 Wallets w/Metamask}


\begin{frame}[fragile]
    \frametitle{Call Metamask}
    This command will cause your browser to connect to Metamask, \\ on TestNet
    \rule{\textwidth}{0.9pt}
    \tiny
    \begin{minted}{javascript}
let alice = await Web3TestNetSmartBchWallet.init();
    \end{minted}
\rule{\textwidth}{0.9pt}
\begin{itemize}
\item   \href{https://docs.smartbch.org/smartbch/testnets}{Official Documentation from SmartBCH}
\item    Network URL: https://moeing.tech:9545
\item    The chain ID is: 0x2711 (i.e. 1001)
\end{itemize}
\end{frame}



  \begin{frame}[fragile]
    \frametitle{Call Regtest Metamask}
    Trying to repeat the steps above for RegTest will throw an alert with instructions to add 
    the network manually with parameters
    \rule{\textwidth}{0.9pt}
    \tiny
    \begin{minted}{javascript}
   await Web3RegTestSmartBchWallet.init();
    \end{minted}
\rule{\textwidth}{0.9pt}
\begin{itemize}
\item  Network Name: SmartBch RegTest
\item  New RPC URL: http://127.0.0.1:8545
\item  Chain ID: 0x2712
\item  Currency Symbol (optional): BCH
\end{itemize}
\end{frame}

\section{SEP-20 Tokens}


\subsection[SEP20 with Minting]{SEP20 with Minting}

\begin{frame}[fragile]
    \frametitle{SEP20 Token genesis with baton}
    After creating a testnet wallet and funding it with come tsBCH... \ 
    the following request creates an SEP20 token with a Baton:
    \rule{\textwidth}{0.9pt}
    \tiny
    \begin{minted}{bash}
curl -X POST https://rest-unstable.mainnet.cash/smartbch/sep20/genesis \
    -H "Content-Type: application/json" \
    -d '{
"walletId":
<your testnet wallet id>,
      "name": "Mainnet Coin",
    "ticker": "MNC",
    "decimals": 8,
    "initialAmount": 10,
"endBaton": false,
"tokenReceiverAddress": <your testnet address>,
"batonReceiverAddress": "<your testnet address>",
}'
    \end{minted}
\rule{\textwidth}{0.9pt}
\end{frame}


\begin{frame}[fragile]
  \frametitle{SEP20 Token genesis with baton}
  the genesis response should return something like this:
  \rule{\textwidth}{0.9pt}
  \tiny
  \begin{minted}{json}
    {
      "tokenId": "0xc216B347d4684fa917dFa52660479935b670564F",
      "balance": {
        "name": "Mainnet Coin",
        "ticker": "MNC",
        "decimals": 0,
        "value": "10000",
        "tokenId": "0xc216B347d4684fa917dFa52660479935b670564F"
      }
    }
  \end{minted}
\rule{\textwidth}{0.9pt}
\end{frame}


\begin{frame}[fragile]
  \frametitle{SEP20 Token Mint Request}
  The following snippet can be used to mint new tokens.
  \rule{\textwidth}{0.9pt}
  \tiny
  \begin{minted}{bash}
curl -X 'POST' \
  'https://rest-unstable.mainnet.cash/smartbch/sep20/mint' \
  -H 'accept: application/json' \
  -H 'Content-Type: application/json' \
  -d '{
  "walletId": <your testnet wallet id>,
  "value": 100000,
  "tokenId": <your token  id>,
  "tokenReceiverAddress": <your receiver address>
}'
\end{minted}
\rule{\textwidth}{0.9pt}
\end{frame}



\begin{frame}[fragile]
  \frametitle{SEP20 Token Mint Response}
  the mint response should return something like this:
  \rule{\textwidth}{0.9pt}
  \tiny
  \begin{minted}{json}
{
  "txId": "0x6a7c03c0ab1c2e170705a7abcbab43cd908cd13f8d416995b0bd6facf48bee42",
  "balance": {
    "name": "Mainnet Coin",
    "ticker": "MNC",
    "decimals": 0,
    "value": "110000",
    "tokenId": "0xc216B347d4684fa917dFa52660479935b670564F"
  }
}
\end{minted}
\rule{\textwidth}{0.9pt}
\end{frame}


%%%%%%%%%%%%%%%%%%%%%%%%%%%%%%%%%%%%%%%%%%%%%%%%%%
%%%%%%%%%%%%%%%%%%%%%%%%%%%%%%%%%%%%%%%%%%%%%%%%%%
%%%%%%%%%%%%%%%%%%%%%%%%%%%%%%%%%%%%%%%%%%%%%%%%%%

\section{State of mainnet.cash}

\begin{frame}{Committed to Adoption}
Mainnet Cash will be here.
% Keep the summary *very short*.
\begin{itemize}
\item
    The project \alert{has some remaining funds to handle maintenance}.
\item
    Currently in a \alert{feature freeze}.
\item
    Seeking \alert{broader adoption}, and usage.
\item
    The project is suitable for small or limited value transactions. \$0.02-\$50
\item
    An independent \alert{security audit} would be in order before the warnings are removed.
\end{itemize}
Warnings aside, mainnet-js was used for the old hop.cash and is in production on \href{https://old.hop.cash}{old.hop.cash}.
\end{frame}

\begin{frame}{The Future}

Mainnet Cash was built to be maintained.
% The following outlook is optional.
\vskip0pt plus.5fill

% Keep the summary *very short*.
\begin{itemize}
\item
    Really conservative and maintenance oriented design choices from the start.
\item
    Extensive investment in testing and a testing harness
\item
    These values are also apparent in supporting libraries.
\end{itemize}


\begin{itemize}
\item
    Outlook
    \begin{itemize}
    \item
    \alert{@bitauth/libauth} has a nice upgrade in the works.
    \item
    There are exiting things are happening with both SmartBCH and BCH
    \end{itemize}
\end{itemize}
\end{frame}

\begin{frame}{Contact}

    
    % The following outlook is optional.
    \vskip0pt plus.5fill
    
    % Keep the summary *very short*.
    \begin{itemize}

       \item More documentation at \href{https://mainnet.cash/tutorial/smartbch.html}{mainnet.cash}
       \item Please report \href{https://github.com/mainnet-cash/mainnet-js/issues}{Bugs on Github}.
       \item Join our \href{https://t.me/mainnetcash}{ mainnet.cash Telegram Channel} for support.
    \end{itemize}

\end{frame}

\end{document}

